\documentclass[12pt]{article}
\title{Physik Abiturn Lernplan}
\author{Jan Kunde}
\begin{document}
\maketitle
\section{Relativitätstheorie}
    \subsection{Konstanz der Lichtegeschwindigkeit}
    \subsection{Problem der Gleichzeitigkeit}
    \subsection{Zeitdilatation und Längenkontraktion}
    \subsection{Relativistische Massenzunahme}
    \subsection{Energie-Masse-Beziehung}
    \subsection{Einfluss der Gravitation auf die Zeitmessung}


\section{Elektrik}
    \subsection{Eigenschaften elektrische Ladungen und ihrer Felder}
    \subsection{Bewegung von Ladungsträgern in elektrischen und magnetischen Feldern}
    \subsection{Elektromagnetische Induktion}
    \subsection{Elektromagnetische Schwingungen und Wellen}

\section{Quantenphysik}
    \subsection{Licht und Elektronen als Quantenobjekte}
    \subsection{WT-Dualismus und Wahrscheinlichkeitsinterpretation}
    \subsection{Quantenphysik und klassische Physik}

\section{Atom-, Kern-, und Elementarteilchenpyhsik}
    \subsection{Atomaufbau}
    \subsection{Ionisierende Strahlung}
    \subsection{Radioaktiver Zerfall}
    \subsection{Kernspaltung und Kernfusion}
    \subsection{Elementarteilchen und ihre Wechselwirkungen}
\end{document}