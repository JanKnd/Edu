
\documentclass{article}
\usepackage{amssymb}
\title{Zusammengesetzte Funktionen}
\author{Jan Kunde}
\date{}
\begin{document}
\maketitle
\begin{center}
    Im folgenden gilt \begin{math} {\displaystyle x \in \mathbb{R} \; \forall f(x)} \end{math}
\end{center}
\raggedright
    \section*{Summenfunktionen:}
        Wenn: \break
        \begin{math}
                {\displaystyle f(x) = u(x) + v(x)}
        \end{math}

        Dann gilt:\break
            \begin{math}
                 {\displaystyle f'(x) = u'(x) + v'(x)}
            \end{math}
        

        \subsubsection*{Beispiel:}
            \begin{math}
                {\displaystyle f(x) = x^3 + ln(x)} \break\break
                {\displaystyle f'(x) = 3x^2+\frac{1}{x}}
            \end{math}

    \section*{Differenzfunktionen:}
    Wenn: \break
    \begin{math}
            {\displaystyle f(x) = u(x) - v(x)}
    \end{math}

    Dann gilt:\break
        \begin{math}
            {\displaystyle f'(x) = u'(x) - v'(x)}
        \end{math}
    

    \subsubsection*{Beispiel:}
        \begin{math}
            {\displaystyle f(x) = x^3 - ln(x)} \break\break
            {\displaystyle f'(x) = 3x^2 - \frac{1}{x}}
        \end{math}

    \section*{Produktfunktionen:}
    Wenn: \break
    \begin{math}
            {\displaystyle f(x) = u(x) \cdot v(x)}
    \end{math}

    Dann gilt:\break
        \begin{math}
            {\displaystyle f'(x) = u'(x) \cdot v(x) + u(x) \cdot v'(x)}
        \end{math}

    \subsubsection*{Beispiel:}
    \begin{math}
        {\displaystyle f(x) = e^x \cdot x^3} \break\break
        {\displaystyle f'(x) = e^x \cdot x^3 + e^x \cdot 3x^2 = e^x(x^3+3x^2)}
    \end{math}

    \section*{Gebrochenrationale Funktionen:}
    Wenn: \break
    \begin{math}
            {\displaystyle f(x) = \frac{u(x)}{v(x)}}
    \end{math}
    
    Dann gilt: \break
    \begin{math}
        {\displaystyle f'(x) = \frac{v(x)\cdot u'(x) - v'(x) \cdot u(x)}{v(x)^2}}
    \end{math}

    Alternativ lässt sich eine gebrochenrationale Funktion \begin{math}
        {\displaystyle f(x) = \frac{u(x)}{v(x)}}
    \end{math} zu \begin{math}
        {\displaystyle f(x) = u(x) \cdot v(x)^{-1} }
    \end{math}
    umformen und mit Produkt- und Kettenregel ableiten.
    \subsubsection*{Beispiel:}
    \begin{math}
        {\displaystyle f(x) = \frac{3x^4}{4x+3}}\break\break
        {\displaystyle f'(x) = \frac{4\cdot 3x^4 - (4x+3) \cdot 12x^3}{(4x+3)^2} = \frac{36x^3\cdot (x+1)}{(4x+3)^2}}
    \end{math}

    \section*{Verkettete Funktionen:}
    Wenn: \break
    \begin{math}
            {\displaystyle f(x) = (u \circ v)(x) = u(v(x))}
    \end{math}

    Dann gilt:\break
        \begin{math}
            {\displaystyle f'(x) = v'(x) \cdot u'(v(x))}
        \end{math}
    

    
    \subsubsection*{Beispiel:}
        \begin{math}
            {\displaystyle f(x) = e^{3x^2}} \break\break
            {\displaystyle f'(x) = 6x \cdot e^{3x^2}}
        \end{math}
    


\end{document}