
\documentclass{article}
\usepackage{amssymb}
\title{Zusammengesetzte Funktionen}
\author{Jan Kunde}
\date{}
\begin{document}
\maketitle
\begin{center}
    Im folgenden gilt $ { x \in \mathbb{R} \; \forall f(x)} $
\end{center}
\raggedright
\section*{Arten zusammengetzter Funktionen und ihre Ableitungen}
    \subsection*{Summenfunktionen}
        Wenn: \break
        $f(x) = u(x) + v(x)$
        \break
        Dann gilt:\break
        $f'(x) = u'(x) + v'(x)$
        \break \break        
        Beispiel: \break \break
        $f(x) = x^3 + ln(x) \break f'(x) = 3x^2+\frac{1}{x}$

    \subsection*{Differenzfunktionen}
    Wenn: \break
    $f(x) = u(x) - v(x)$
    \break
    Dann gilt:\break
    $f'(x) = u'(x) - v'(x)$
    \break \break
    Beispiel: \break \break
    $f(x) = x^3 - ln(x) \break f'(x) = 3x^2 - \frac{1}{x}$

    \pagebreak
    
    \subsection*{Produktfunktionen}
    Wenn: \break
    $f(x) = u(x) \cdot v(x)$
    \break
    Dann gilt:\break
    $f'(x) = u'(x) \cdot v(x) + u(x) \cdot v'(x)$
    \break \break
    Beispiel: \break \break
    $f(x) = e^x \cdot x^3 \break
    f'(x) = e^x \cdot x^3 + e^x \cdot 3x^2 = e^x(x^3+3x^2)$
    


    \subsection*{Gebrochenrationale Funktionen}
    Wenn: \break
    $f(x) = \frac{u(x)}{v(x)}$
    \break
    Dann gilt: \break
    $f'(x) = \frac{v(x)\cdot u'(x) - v'(x) \cdot u(x)}{v(x)^2}$
    Alternativ lässt sich eine gebrochenrationale Funktion 
    $f(x) = \frac{u(x)}{v(x)}$ zu $f(x) = u(x) \cdot v(x)^{-1}$
    umformen und mit Produkt- und Kettenregel ableiten.
    \break \break
    Beispiel: \break \break
    $f(x) = \frac{3x^4}{4x+3}\break f'(x) = \frac{4\cdot 3x^4 - (4x+3) \cdot 12x^3}{(4x+3)^2} = \frac{36x^3\cdot (x+1)}{(4x+3)^2}$

    \subsection*{Verkettete Funktionen}
    Wenn: \break
    $f(x) = (u \circ v)(x) = u(v(x))$
    \break
    Dann gilt:\break
    $f'(x) = v'(x) \cdot u'(v(x))$
    \break \break
    Beispiel: \break \break
        $f(x) = e^{3x^2} \break f'(x) = 6x \cdot e^{3x^2}$

    \pagebreak

\section*{Untersuchung zusammengesetzte Funktionen}
\subsection*{Bestimmung von Definitionsmengen}
Außer bei gebrochenrationalen Funktionen und Funktionen die als Term einen $ln$ oder $tan$
enthalten ist die Definitionsmenge aller abiturrelevanten Funktionen: $D = \mathbb{R}$
    \subsubsection*{Definitionsmengen gebrochenrationaler Funktionen}
    Gebrochenrationale Funktionen weisen Definitionslücke an den Nullstellen des Nenners auf.
    \break \break
    Beispiel: \break \break
    $f(x) = \frac{3x^5+2x}{x^2-3} \break \break
    x^2-3 = 0 \quad |+3 \break
    x^2 = 3 \break
    x_{1,2} = \pm\sqrt{3} \break \break
    D = \mathbb{R} \setminus \{ -\sqrt{3},\sqrt{3} \}
    $

    \subsection*{Definitionsmengen bei ln-Funktionen}
    Der natürliche Logarithmus ist nur für Eingabewerte größer als Null definitiert.
    Besteht eine Funktion aus einer Verkettung aus natürlichem Logarithmus und einer weiteren Funktion,
    umfasst die Definitionsmenge nur die X-Werte, für die der innere Teil der Funktion größere Werte als Null annimmt.
    Zur Bestimmung der Definitionsmenge muss also die Ungleichung $v(x) > 0$ gelöst werden, wobei $v(x)$ der innere Teil der zusammengesetzten ln-Funktion ist.
    \break \break
    Beispiel:\break \break
    $
    f(x) = ln(x+3) \break \break
    x+3 > 0 \quad |-3 \break
    x > -3 \break \break
    D = \{x \in \mathbb{R}: x > -3\}
    $
    \end{document}