\documentclass[12pt]{article}
\title{EVA-Aufgaben 17.01.2024}
\author{Jan Kunde}
\begin{document}
\maketitle
\section{Worauf ist die Entstehung der Vereinten Nationen zurückzuführen? }
Die Entstehung der Vereinten Nationen ist auf das scheitern des Volkerbundes, dem vorherigen Konstrukt
zur Friedensicherung unter den Nationen, angesichtes des zweiten Weltkrieges zurückzuführen. Im Jahr 1945 entschloss
man sich daher die Vereinten Nationen mit anfangs 51 Mitgliedsstaaten zu gründen,
 deren Anzahl bis heute auf 193 gestiegen ist.
\section{Welche Ziele verfolgen die Vereinten Nationen? Welche Grundsätze gelten dabei?}
\subsection{Ziele der Vereinten Nationen}
\begin{enumerate}
    \item Wahrung des Weltfriedens, der Internationalen Sicherheit und Unterdrückung internationaler Streitigkeiten
    \item Freundschaftliche Beziehungen zwischen Nationen herstellen und festigen.
    \item Internationale Zusammenarbeit bei der Lösung vielfaltiger Probleme
    \item Epizentrum dieser Internationaler Zusammenarbeit zu sein
\end{enumerate}
\subsection{Dabei geltende Grundsätze}
\begin{enumerate}
    \item Souveräne Gleichheit der Mitgliedsstaaten
    \item Jedes Mitglied befolgt die geltenden Grundsätze
    \item Alle internationalen Streitigkeiten werden auf friedlichem Wege beigelegt.
    \item Mitglieder unterlassen mit den Grundsätzen unvereinbare Androhungen und Anwendungen von Gewalt
    \item Mitglieder unterstützen die Vereinten Nationen bei jeder Maßnahme die mit den Grundsätzen vereinbar ist
    \item Die Grundsätze geben den Vereinten Nationen nicht das Recht in innere Vorgänge der Mitgliedsstaaten einzugreifen
\end{enumerate}
\section{Mögliche Probleme, die sich aus der UNO-Charta ergeben könnten}
\begin{itemize}
    \item Handlungsfähigkeiten der Vereinten Nationen beschränken sich auf friedliche Mittel,
was die Macht der Vereinten Nationen gegenüber ihrer Mitglieder stark beschränkt 
\end{itemize}
\end{document}