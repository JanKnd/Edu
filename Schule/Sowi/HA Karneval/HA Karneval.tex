\documentclass{article}
\author{Jan Kunde}
\date{}
\title{Inwiefern liefert die UN einen legitimen Beitrag zur Friedenssicherung in der Ukraine?}
\begin{document}
\maketitle

\noindent Nach erlangen umfangreicher Informationen über den Aufbau 
der Vereinten Nationen stellt sich die Frage, inwiefern diese einen effizienten und legitimen Beitrag zur Friedenssicherung
in der Ukraine leisten kann. Diese Frage werde ich versuchen im folgenden differenziert zu Beantworten.
\break
\break
Richtet man seinen Blick auf den UN-Sicherheitsrat, so sticht ein dauerhaftes Mitglied gleich ins Auge. Russlands Veto-Recht im Sicherheitsrat könnte für die
Durchsetzbarkeit von Entscheidungen innerhalb der UN im Zusammenhang mit dem Ukrainekonflikt insbesondere bezüglich Peace-Enforcement aktionen eine enorme Hürde bedeuten
Jedoch könnte die UN einen beitrag zur Wiederherstellung des Friedensleistet, indem sie als Verhandlungsraum aggiert.
Des Weiteren lässt sich argumentieren, dass die Vereinten Nationen den Frieden in der Ukraine durch den Aspekt der Souveränität "unterstützen", allein dadurch, dass die Ukraine von ihr
als eigenständiger Staat anerkannt wird.
Außerdem lässt sich anmerken, dass die UN über weitaus mehr Handlungsmöglichkeiten verfügt als bloß militärische Peace-Keeping und -Enforcement Aktionen. So ist ihre Handlungsfähigkeit
vorallem im Hinblick auf humanitäre Hilfe nicht vollständig eingeschränkt, da diese nicht der direkten Bestimmung des Sicherheitsrates unterliegt.
\break
\break
Zusammenfassend lässt sich sagen, dass es sich für die Vereinten Nationen schwierig gestalten dürfte direkt zur Friedenssicherung und -schaffung beizutragen, solange Russland das 
Veto-Recht im Sicherheitsrat inne hat. Jedoch kann die UN dennoch insbesondere die lokale Bevölkerung auf zivilem Wege unterstützen und vermittelnd zwischen den Pateien aggieren. 
\end{document}