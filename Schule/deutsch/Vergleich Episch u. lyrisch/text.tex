\documentclass{article}
\author{Jan Kunde}
\date{}
\title{Epik und Lyrik}

\begin{document}
\raggedright
\maketitle
\subsection*{Robert Musil: Die Verwirrung der Zöglings Törleß}
    Die epische Erzählung\textit{\textit{ "Die Verwirrung des Zöglings Törleß" }}von Robert Musil aus dem Jahr 1906 thematisiert die comming-of-age-Geschichte des Schülers Törleß.
    Speziell im gegeben ausschnitt versucht Törleß seinen Gedanken ausdruck zu verleihen, was sich jedoch für ihn auf Grund mangelnden Wortschatzes für ihn als schwierig erweist.
    \break\break
    Der Ausschnitt aus Musils Roman folgt einer iterativen Erzählstruktur. Immer wieder
    kommt der personale Erzähle, welcher aus Sicht Törleß spricht auf die selben Gedanken zurück
    und erweitert diese bei jeder Iteration (vgl. Zeile 4 ff.). Dieses Erzählschema
    armt den natürlich menschlichen Denkprozess nach und verdeutlicht auf diese Weise,
    das es um die direkten Gedanken des Erzählers handelt. Außerdem wird hieran Törleß selbst beschriebene
    Ungeübtheit mit Sprache (vgl. Zeile 1) deutlich.
    \break
    \noindent
    Törleß sieht die\textit{ "Dinge in zweierlei Gestalt" }(Zeile 4 f), wobei mit\textit{ "Dingen" }höchstwarscheinlich
    sein inneres Wesen gemeint ist. Törleß teilt diese\textit{ "Dinge" }in einen aus Gedanken bestehenden Teil
    und nutzt das Motiv der Dunkelheit um die andere Seites dieser\textit{ "Dinge" }zu beschreiben, was ihre
    mysteriösität und unbeschreiblichkeit hervorhebt.\break
    Törleß nutzt den Kontrost zwischen der Dunkelheit und dem Aufleben\textit{ "unter einem anderen Lichte" }
    (Zeile 9) um zu zeigen, dass es sich bei dieser mysteriösen Seite der Dinge zwar um etwas handelt,
    das sich nicht verbalisieren und beschreiben lässt aber für ihn dennoch einen beinahe transzendenten
    Sinn ergibt.
    \break\break
    Abschließend lässt sich sagen, dass Törleß im gegebenen Romanauszug versucht seinen Erkenntnissen
    hinsichtlich seiner Selbst Ausdruck zu verleihen und dabei kleinschrittig seine Gedankengänge darstellt.
    Der Ausschnitt zeigt deutlich die Grenzen von Sprache auf.
\pagebreak
\subsection*{Gottfried Benn: Ein Wort}
    Das Gedicht \textit{"Ein Wort"} von Gottfried Benn aus dem Jahr 1941 lässt sich der Epoche des Expressionismus
    zuordnen und thematisiert den aus Chiffren erwachsenden Sinn.

    Benns Gedicht setzt sich aus zwei Quartetten zusammen. Beim Reimschema des Gedichtes handelt es sich im
    stetige Kreuzreime welche die Verworrenheit des Inhalts wieder spiegeln. Das Metrum des Gedichts ist ein
    vierhebiger Jambus mit abwechselnd männlichen und weiblichen Kadenzen.

    Benn beginnt sein Gedicht mit der Klimax \textit{"Ein Wort - Ein Satz"} und ergänzt diese gleich darauf mit einer weiteren Klimax\textit{"aus Chiffren steigen erkanntes Leben, jäher Sinn"} (Vers 1).
    Bereits hier wird die hohe Bedeutung deutlich die Benn der Sprache zumisst.
    Die die Personifizierung der Sonne und der Sphären(vgl. Vers 3), die sich ihm hin wenden, stellt Benn das Wort in den Mittelpunkt des Universums.
    So lässt sich erkennen, dass Benn Chiffren als etwas ansieht aus dem alles entstehen kann.
    Auch die zweite Strophe beginnt mit der Anapher zum Beginn der Ersten Strophe \textit{"Ein Wort"} und einer direkt darauf folgenden Klimax \textit{"ein Glanz, ein Flug, ein Feuer, ein Flammenwurf, Sternenstrich}
    Diese Klimax kontrastiert Benn mit der Dunkelheit (vgl Vers 7) und beendet das Gedicht mit dem Vers \textit{"im leeren Raum um Welt und Ich} (Vers 8). Damit zeigt er, dass das Wort unserer Welt
    ihr "Licht" verleiht also ein essenzieller Bestandteil des Daseins ist. 

\subsection*{Fazit}
    Zum Ende lässt sich herausstellen, dass in Musils Text dem Wort und der Sprache eine weitaus weniger Tragende Bedeutung für die Welt zukommt als in Benns Gedicht.
    Wo bei Benn das Wort metaphorisch als Mittelpunkt des Sonnensystems dargestellt wird, scheint es bei Musil benahe nur limitierendes Element bei der Kommunikation zu sein.
\end{document} 