\documentclass{article}
\author{Jan Kunde}
\date{}
\title{Inwiefern liefert die UN einen legitimen Beitrag zur Friedenssicherung in der Ukraine?}
\begin{document}
\maketitle

\noindent Nach erlangen umfangreicher Informationen über den Aufbau 
der Vereinten Nationen stellt sich die Frage, inwiefern diese einen effizienten und legitimen Beitrag zur Friedenssicherung
in der Ukraine leisten kann. Diese Frage werde ich im folgenden differenziert Beantworten.
\break
\break
Bei Auseinandersetzung mit dieser Frage wird gleich zu Begin ein Problem hinsichtlich der Durchsetzbarkeit von Maßnahmen zur Friedensschaffung und Friedenssicherung deutlich.
Russlands dauerhafter Platz im Sicherheitsrat und das damit einhergehende Veto-Recht machen eine Einigung über militätische Maßnahmen praktisch unmöglich, da Russland wohl kaum Anträgen zum
eigenen Nachteil stattgeben würde.
Anderer Seits ergibt sich auf Grund der so genannten "Unity for Peace" UN Resolution 377 von 1950, eine weiter Chance, denn diese legt fest, dass bei anhaltender Uneinigkeit
des Sicherheitsrates die Generalversammlung dazu befugt ist Handlungsempfehlungen, auch militärischer Natur für die UN-Gemeindschaft auszusprechen. Außerdem währe es der UN möglich humanitäre
Hilfe im Kriegsgebiet zu leisten, da solche Beschlüsse nicht die Zustimmung des Sicherheitsrates erfordern. 
\break
Des Weiteren lässt sich argumentieren, dass militätische Eingriffe in den Konflikt durch die UN eine Positionierung dieser zur Streitfrage vorraussetzen würde. Dies hat zur Folge,
dass egal auf welche Seite des Konfliktes sich die Vereinten Nationen positionieren stehts eine der Konfliktparteien bei militätischem Eingriff ihre Souveränität als kompromitiert
ansähe. Russland, da es die Ukraine als Teil des eigenen Staates ansieht und die Ukraine, da sie sich selbst als eigenständige Nation betrachtet.
\break 
Anderer Seits müsste sich ein Beitrag zur Friedenssicherung nicht zwingend durch konkrete Eingriffe in das Kriegsgeschehen kennzeichnen. Auch das Agieren der UN als Verhandlungsraum und Vermittler zwischen
den Konfliktparteien könnte einen signifikanten Beitrag leisten.
\break
\break
Meiner Meinung nach ist es für die UN so gut wie unmöglich militätisch in den Konflikt einzugreifen, aufgrund Russlands Veto-Recht und der Tatsache dass es sich bei Empfehlungen der
Generalversammlung eben nur um Empfehlungen handelt und keinensfalls um verbindliche Vorgaben wie es Beschlüssen des Sicherheitsrates wären. Die UN müsste sich also auf eine Rolle als
Vermittler oder humanitären Helfer beschränken, was meiner Ansicht nach dennoch zu einer deutlichen Verkürzung des Konfliktes und verbesserten Lebenssituation für die im Kriegsgebiet
lebenden Menschen führen könnte.

\end{document}