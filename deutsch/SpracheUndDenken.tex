\documentclass{article}
\title{Sprache und Denken \\ David Crystal}
\author{Lina Schrage, Jan Kunde}

\begin{document}
\maketitle
Zur Beantwortung der Frage wie eng die Verbindung von Sprache und Denken sei betrachtet Crystal die hypothetischen möglichkeiten als Spektrum zwischen den beiden Extremen,
der Identität von Sprache und Denken, und Sprache und Denken als zwei verschiedene und dennoch von einander abhängige Dinge. Er vermutet, dass die Wahrheit zwischen diesen beiden Polen liegt. Nach der Hypothese der Abhängigkeit von Sprache und Denken bestehen laut Crystal zwei Möglichkeiten. Einerseits sei es möglich, dass zunächst die Fähigkeit zu Denken erlangt wird und erst danach die des Sprechens, andererseits bestehe die Möglichkeit, dass die Sprache die Art und Weise vorgibt in der ein Mensch denken kann. Im Bezug auf den Spracherwerb scheint jedoch erstere Möglichkeit warscheinlicher, da Kinder bereits vor dem Spracherwerb kognitive Fähigkeiten entwickeln. Eine weitere These besagt, dass Sprache und Denken zwar voneinander abhängig jedoch nicht identisch seien. Dies lässt sich beispielsweise anhand der Erinnerungen an eine Bewegungsabfolge bei Spiel und Sport, wobei keine Sprache verwendet wird beweisen. Crystal schreibt, dass wenn man Sprache und Denken als voneinander abängig betrachet, man erkennt, dass die Sprache ein Teil des Denkprozesses sei und man gleichzeitig das Denken als notwendige Vorraussetzung für das Sprachverständnis ansehe. Er beendet seine These mit der Aussage, dass sowohl Sprache als auch Denken wesentlich seien.
\end{document}
